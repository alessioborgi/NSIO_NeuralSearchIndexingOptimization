\documentclass{article}


\usepackage{PRIMEarxiv}
\usepackage{todonotes}
\usepackage[utf8]{inputenc} % allow utf-8 input
\usepackage[T1]{fontenc}    % use 8-bit T1 fonts
\usepackage{hyperref}       % hyperlinks
\usepackage{url}            % simple URL typesetting
\usepackage{booktabs}       % professional-quality tables
\usepackage{amsfonts}       % blackboard math symbols
\usepackage{nicefrac}       % compact symbols for 1/2, etc.
\usepackage{microtype}      % microtypography
\usepackage{lipsum}
\usepackage{subfiles}
\usepackage{fancyhdr}       % header
\usepackage{graphicx}       % graphics
\usepackage{algpseudocodex}
\usepackage{algorithm}
\usepackage{amsmath, amssymb}
\graphicspath{{media/}}     % organize your images and other figures under media/ folder

%Header
\pagestyle{fancy}
\thispagestyle{empty}
\rhead{ \textit{ }} 

% Update your Headers here
\fancyhead[LO]{Neural Search Indexing Optimization: Integrating Augmentation and PEFT for Efficient Retrieval}
% \fancyhead[RE]{Firstauthor and Secondauthor} % Firstauthor et al. if more than 2 - must use \documentclass[twoside]{article}

  
%% Title
\title{Neural Search Indexing Optimization: Integrating Augmentation and PEFT for Efficient Retrieval}

\author{
  Alessio Borgi, Eugenio Bugli, Damiano Imola \\
  1952442, 1934824, 2109063 \\
  Sapienza Università di Roma \\
  \texttt{\{borgi.1952442, bugli.1934824, imola.2109063\}@studenti.uniroma1.it} \\
}

\begin{document}
\maketitle

\begin{abstract}
In this project report we present a novel Differentiable Search Indexing (DSI) approach. Unlike traditional contrastive learning-based dual encoders, this architecture maps a query to its relevant document identifier (docid) during inference, enabling efficient retrieval with standard model inference and optional beam search for ranking. We explore advanced indexing methods, including semantic and syntactic data augmentation techniques like stopword removal, Num2Text transformation, and POS-MLM augmentation, to enhance dataset quality. Our model innovations feature dynamic pruning for efficient training, parameter-efficient fine-tuning via LoRA, and memory optimization through QLoRA with 4-bit quantization.

\end{abstract}

% keywords can be removed
\keywords{DSI \and POS-MLM \and Dynamic Pruning \and LoRA \and QLoRA \and AdaLoRA \and ConvoLoRA}

\section{Introduction} \subfile{sections/intro.tex}
\section{Dataset} \subfile{sections/data.tex}
\begin{figure*}
  \centering
  \includegraphics[width=\textwidth]{figs/dataset.png}
  \caption{Data Augmentation Pipeline.}
  \label{fig:fig1}
\end{figure*}
\section{Model} \subfile{sections/model.tex}
\section{Training and Results} \subfile{sections/training.tex}



\section{Conclusion}
In this work, we enhanced the Differentiable Search Index (DSI) by integrating novel data augmentation techniques and parameter-efficient fine-tuning (PEFT) methods to improve retrieval effectiveness and computational efficiency. Our augmentation strategies, including Num2Word transformation, stopword removal, and POS-MLM augmentation, enriched semantic representations, reducing noise and improving contextual understanding. 

In parallel, we leveraged PEFT techniques such as LoRA, QLoRA, AdaLoRA, and ConvLoRA to optimize training and inference, reducing memory consumption while preserving retrieval accuracy. 

Future research will focus on extending these optimizations to larger datasets, refining augmentation techniques, and exploring further parameter-efficient methodologies to enhance retrieval capabilities in real-world applications.




%Bibliography
\bibliographystyle{unsrt}  
\bibliography{references}  

\appendix
\section{Semantically Structured Docids Generation Algorithm}
\label{sec:semanticids}
\subfile{sections/ssdocids.tex}

\section{DSI Overall Architecture}
\subfile{sections/dsi.tex}

\section{Fine-Tuning Results: Validation Loss}
\label{sec:metrics}
In the following figure, we present the validation loss for the different fine-tuning techniques across a sample of 5 epochs. We limited to this number due to computational constraints.
\begin{figure}[H]
  \centering
  \includegraphics[width=0.8\textwidth]{figs/metrics.png}
  \caption{Validation Loss for different fine-tuning techniques across 5 epochs.}
  \label{fig:metrics}
\end{figure}




\end{document}
