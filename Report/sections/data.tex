Given the MS Marco Dataset, we need to perform some preprocessing in order to create an association between queries and docids.
This Dataset is composed by 100k passages text which are partitioned into Training, Validation and Test. Each partition is composed by lists of passages texts and related queries.
We create a set $\mathcal{U} = \mathcal{O} \cup \mathcal{P}$, where $\mathcal{O}$ is the set composed by each segment (passage text) of the MS Marco Dataset, while $\mathcal{P}$ is the set of pseudo-queries from each segment of $\mathcal{O}$, which are artificially generated by a model (docT5query).
After this procedure, we need to use segments belonging to $\mathcal{U}$ as queries with a ranker. This procedure gives us the first $\mathcal{k}$ ranked documents that have relevant information with respect to the segment.
By doing this, we check if a segment has enough information to represent the document and can be used as a suitable query.